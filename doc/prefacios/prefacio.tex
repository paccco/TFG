\thispagestyle{empty}

\begin{center}
{\large\bfseries App diario de ejercicio físico}\\
\end{center}
\begin{center}
Francisco de Asís Carrasco Conde\\
\end{center}

\vspace{0.7cm}

\noindent\textbf{Resumen}\\  
	
La motivación de este TFG es la falta de aplicaciones sencillas para personas que realizan ejercicio de forma informal. Muchas apps actuales resultan complejas, poco inclusivas y/o son de pago.	
	
En este TFG se va a desarrollar una aplicación que ayude a las personas a realizar ejercico físico de forma controlada y a monitorizar las marcas que se van alcanzando. Se plantea desarrollar una app móvil para android y iOS que facilite la planificación/monitorización del entrenamiento, permitiendo compartir rutinas de entrenamientos entre usuarios. El diseño tendrá en cuenta principios de accesibilidad para mejorar la experiencia de uso.

Las herramientas usadas para su desarrollo son Express.js(Node.js) para el servidor, mySQL para la base de datos del servidor, Flutter(Dart) para la aplicación movil y SQLite para el almacenamiento local en el dispositivo móvil. Para la planificación durante el desarrollo se ha usado Jira y para generar esta documentación Latex.

\newpage

\begin{center}
	{\large\bfseries Same, but in English}\\
\end{center}
\begin{center}
	Francisco de Asís Carrasco Conde\\
\end{center}
\vspace{0.7cm}

\noindent\textbf{Abstract}\\
The motivation for this TFG is the lack of simple applications for people who exercise informally. Many current apps are complex, not inclusive and / or paid

In this TFG, an application will be developed that helps people to perform physical exercise in a controlled way and to monitor the marks that are being reached. It is proposed to develop a mobile app for android and iOS that facilitates the planning / monitoring of training, allowing the sharing of training routines among users. The design will take accessibility principles into account to improve the user experience.

The tools used for its development are Express.js ( Node.js ) for the server, mySQL for the server database, Flutter ( Dart ) for mobile application and SQLite for local storage on mobile device. Jira has been used for development planning and to generate this Latex documentation.

\chapter*{Agradecimientos}

Me gustaría dar agradecimientos a mis familiares, en especial a mis padres y a mi hermana por apoyarme durante todo el transcurso de mi carrera hasta llegar hasta aquí. Ha sido un camino largo con sus picos y sus bajadas pero siempre estuvieron ahí.

También me gustaría dar agradecimientos a los compañeros que hice durante este camino, gracias a los cuales nos nutrimos mutuamente para crecer tanto académica como personalmente.

\chapter*{}
\thispagestyle{empty}

\noindent\rule[-1ex]{\textwidth}{2pt}\\[4.5ex]

Yo, \textbf{Francisco de Asís Carrasco Conde}, alumno de la titulación TITULACIÓN de la \textbf{Escuela Técnica Superior
de Ingenierías Informática y de Telecomunicación de la Universidad de Granada}, con DNI 26301846N, autorizo la
ubicación de la siguiente copia de mi Trabajo Fin de Grado en la biblioteca del centro para que pueda ser
consultada por las personas que lo deseen.

\vspace{6cm}

\noindent Fdo: Francisco de Asís Carrasco Conde

\vspace{2cm}

\begin{flushright}
Granada a 13 de Junio de 2025.
\end{flushright}

\thispagestyle{empty}

\noindent\rule[-1ex]{\textwidth}{2pt}\\[4.5ex]

D. \textbf{María José Rodríguez Fórtiz}, Profesora del Departamento de Lenguajes y Sistemas Informáticos

\vspace{0.5cm}

\textbf{Informo:}

\vspace{0.5cm}

Que el presente trabajo, titulado \textit{\textbf{EjercitaTec}},
ha sido realizado bajo mi supervisión por \textbf{Francisco de Asís Carrasco Conde}, y autorizo la defensa de dicho trabajo ante el tribunal
que corresponda.

\vspace{0.5cm}

Y para que conste, expiden y firman el presente informe en Granada a Junio de 2025.

\vspace{1cm}

\textbf{El/la director(a)/es: }

\vspace{2cm}

\noindent \textbf{María José Rodríguez Fórtiz}
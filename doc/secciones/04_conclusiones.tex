\chapter{Conclusiones}

\section{Objetivos}

Los objetivos perseguidos en este trabajo han sido los previamente mencionados en la sección objetivos en la introducción, los cuales son:
\begin{itemize}
	\item Analizar algunas apps del mercado, así como sus características, qué ofrecen, su costo para el usuario y las valoraciones de los usuarios finales, para aclarar qué es lo que buscan los usuarios en este tipo de apps y considerar esas necesidades en el software a desarrollar.
	\item Investigar sobre qué herramientas usar para la implementación de la base de datos, backend, frontend, y técnicas y metodologías para dar un desarrollo de calidad al software y para ayudar a la sostenibilidad del sistema en el tiempo.
	\item Conocer y aprender a usar herramientas actuales y punteras para el desarrollo de aplicaciones móviles. 
	\item Desarrollar una aplicación multiplataforma que cubra todos los requisitos necesarios para dar soporte a la planificación y realización de entrenamientos y monitorización de rutinas de ejercicio físico. 
	\item Tratar de obtener un software lo más accesible posible.
\end{itemize}

Analizando cada objetivo individualmente:
\begin{enumerate}
	\item \textbf{Analizar algunas apps del mercado, así como sus características, qué ofrecen, su costo para el usuario y las valoraciones de los usuarios finales, para aclarar qué es lo que buscan los usuarios en este tipo de apps y considerar esas necesidades en el software a desarrollar}: en el capítulo 2 se trata el estado el arte, en el cuál se mencionan las apps examinadas y las ventajas/desventajas de cada una respecto al software desarrollado, por tanto este objetivo se da por cumplido.
	\item \textbf{Investigar sobre qué herramientas usar para la implementación de la base de datos, backend, frontend, y técnicas y metodologías para dar un desarrollo de calidad al software y para ayudar a la sostenibilidad del sistema en el tiempo}: durante la realización del capítulo de estado del arte se examinaron distintas herramientas a usar, evidentemente no se iban a usar todas las examinadas pero aún sin haber puesto en práctica alguna de las tecnologías examinadas se aprendió sobre su uso y situaciones en las que emplearlas. Sobre las tecnologías previamente ya conocidas, se profundizó más sobre su uso. Este objetivo también se ha cumplido.
	\item \textbf{Conocer y aprender a usar herramientas actuales y punteras para el desarrollo de aplicaciones móviles}: este apartado esta muy relacionado con el anterior. Sin duda lo que más se ha mejorado es el trabajo con la metodología SCRUM, se ha mejorado su entendimiento y refinado conceptos que se creían aprendidos. cabe resaltar la soltura desarrollada con el framework Express.js(Node.js), que permitió el desarrollo de una API de forma rápida y fácil. Se cumplió este objetivo.
	\item \textbf{Desarrollar una aplicación multiplataforma que cubra todos los requisitos necesarios para dar soporte a la planificación y realización de entrenamientos y monitorización de rutinas de ejercicio físico}: cabe decir que no se han desarrollado todas las funcionalidades establecidas desde un principio en el backlog, pero se ha cumplido este objetivo, dado que el software da soporte a la planificación con el calendario, a la realización dado que guarda las marcas y ayuda a llevar los descansos del usuario y monitorización porque podemos observar el desarrollo del usuario a lo largo del tiempo mediante las gráficas.
	\item \textbf{Tratar de obtener un software lo más accesible posible}: No se ha cumplido por completo, dado que algunos aspectos como tener iconos significativos en todos los accesos, se puede hacer complejo dado el tipo de app que es. No obstante, el usuario siempre controla el flujo de la app, el software nunca hace cambios bruscos de pantalla ,ni destellos molestos además de poseer una paleta de colores que relaja al usuario.
\end{enumerate}

\section{Valoración personal}

Hacer este TFG ha supuesto un gran desafio tanto académico como personal. Me he enfrentado a problemas de desarrollo y diseño reales de aplicaiones completas, como pueden ser fallos en la planificación, decidir si descartar una funcionalidad o no. Lo que más se ha refinado es la forma de trabajar usando metodología ágil SCRUM, que ya se había trabajado previamente en el grado, pero me ha permitido profundizar aún más, mejorando mi capacidad de organización, priorización de tareas y adaptación a cambios. Además, he tomado conciencia de la importancia del diseño centrado en el usuario y la accesibilidad.

También me ha ayudado a desarrollar mi comunicación, tanto usando esta documentación como ficha técnica como expresando datos en la propia app.

\section{Objetivos de desarrollo sostenible}

La integración de principios de desarrollo sostenible en la creación y evolución de la app no solo garantiza su viabilidad a largo plazo, también asegura que el impacto de la aplicación sea positivo tanto para los usuarios como para el entorno. A medida que el mercado y las tecnologías evolucionen, la app puede adaptarse y crecer.

Para ello habrá que centrarse en lo que respecta a la app:
\begin{itemize}
	\item Optimización para mayor rendimiento y eficiencia energética, se ha tenido en cuenta la optimización de consultas a la base de datos y el uso de algoritmos eficientes para el procesamiento de datos de entrenamientos. La app también gestiona el uso de la memoria interna de forma eficiente,
	\item Sostenibilidad del ciclo de vida software, para el mantenimiento y actualizaciones a largo plazo. La app está diseñada con un enfoque modular y flexible, lo que permite una fácil adaptación a cambios futuros y a nuevas tecnologías. La metodolgía SCRUM permite optimizar el desarrollo de una app con un tiempo limitado en términos de calidad.
	\item Accesibilidad y diseño inclusivo, en esta app se ha tenido en cuenta en la medida de lo posible, siendo compatible con lectores de pantallas y usandi diseños que favorecen la inclusión.
	\item Uso responsable de datos y privacidad. La app cuando pide datos al usuario los asocia a su nombre y no hacia una persona en concreto, además estos datos nunca son subidos a la nube. Por tanto, se cumple esto bastante bien.
\end{itemize}

\section{Futuro empresarial}

Si se desease convertir esto en un modelo de negocio, habría varias opciones entre ellas:

\begin{itemize}
	\item Dar una versión gratuita con funcionalidades limitadas y otra versión de pago que de funcionalidades más completas y refinadas
	\item Se podría limitar el número de elementos en la nube por usuario y si este quisiese ampliar dicho número que pagué por esa ampliación 
	\item Hacer que la app la puedan usar gimnasios permitiendo que los misms suban rutinas como si fueran un usuario más y ellos serían los que pagan por estar en la nube
\end{itemize}

\section{Trabajos futuros}

En el backlog se han quedado todas las funcionalidades relacionadas con la IA y el smartwatch, las cuales son:

\begin{itemize}
	\item[\textbf{SCRUM-10}] Valorar el entrenamiento en base a la marca actual y la meta del usuario
  	\item[\textbf{SCRUM-11}] Monitorizar pulso en tiempo real
	\item[\textbf{SCRUM-12}] Medir pulso en reposo y compararlo con datos de ejercicios
	\item[\textbf{SCRUM-13}] Avisar de anomalías en el pulso de forma suave
	\item[\textbf{SCRUM-14}] Obtener calorías quemadas
	\item[\textbf{SCRUM-15}] Comprobar el equilibrio nervioso del usuario
	\item[\textbf{SCRUM-17}] Conectar con la IA para iniciar diálogo
	\item[\textbf{SCRUM-19}] Resumir datos
	\item[\textbf{SCRUM-21}] Medir SpO2
	\item[\textbf{SCRUM-22}] Interpretar constantes
\end{itemize}

Todas estas se podrían implementar para versiones futuras. También se podría añadir videos explicativos a los ejercicios, añadir más parámetros de medición a los ejercicios, etc.
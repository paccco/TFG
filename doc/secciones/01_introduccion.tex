\chapter{Introducción}

\section{Justificación}

Primero de todo,¿que me lleva a hacer este TFG? Para empezar, es verdad que hay muchas apps en las stores que son para usuarios de gimnasio, para hacer más fácil el seguimiento y evolución, no obstante, dichas aplicaciones suelen medir parámetros de forma muy general. Me explico, suelen medir calorías quemadas basadas en las pulsaciones (si tienes un reloj inteligente que lo permita) pero este tipo de mediciones de calorías suelen ser poco exactas, me baso en el siguiente estudio \cite{jerath2023future}, como añadido a esto no suele ser un dato de mucha relevancia al hacer datos ejericicios de fuerza. Además la gente que suele hacer deporte de manera más informal o por hobbie, no suele tener los conocimientos o aparatos para hacer mediciones de calidad e interpretarlo correctamente. 

Otra problemática en el resto de apps, suele ser la falta de accesibilidad, los scrolls abundan, eventos no controlados por el usuario, colores confusos para daltónicos, a veces hay ausencias de iconos y un amplio etcétera. También no suele existir un apartado para ver entrenamientos recomendados por otros, como si fuera un foro, la gente suele buscarlo en redes sociales o videos que se encuentran en la red, muchas veces en estos videos se ponen a dar rodeos para rascar más tiempo, así es como el usuario pierde tiempo para que a lo mejor no sea el entrenamiento que el andaba buscando.

Me veo en la necesidad de hacer esta app para dotar de una herramienta simplificada, fácil de manejar, que permita compartir información entre usuarios de forma eficaz y accesible. Simplificada porque la información que manejo se trata de una forma fácil e intuitiva, con un poco de experiencia en el deporte se sabe que significa cada parámetro y la que no es tan fácil de interpretar, la interpreta el sistema por el usuario. Fácil de manejar, dado que el usuario solo se tendrá que preocupar de por ejemplo, contar cuantas veces levanto una pesa o cuando empiezo y acabo de correr. Compartir información eficazmente, porque en una sección de la app habrá una parte formato red social, que permitirá tanto buscar usuarios que comparten entrenamientos, como los propios entrenamientos en si, acompañados de su descripción en la que el creador da una breve información acerca del entrenamiento. Accesible, se evitará en todo lo posible usar elementos que perjudiquen a otros usuarios de diversas capacidaddes, entre dichos elementos incluye scrollables y más elementos.

Una vez explicado esto, ¿como se le dará soporte a los usuarios? De la siguiente forma, midiendo los parámetros que pida el usuario que pueden ser repeticiones/peso/distancia/tiempo se pueden medir de forma individual o al mismo tiempo. También guardando la cantidad de series de un ejercicio realizadas en un entrenamiento con sus respectivos parámetros. Contabilizandole al usuario el tiempo descansado, para facilitar el correcto entrenamiento del usuario. También si el usuario lo solicita se le enseñará un resumen de su rendimiento en cada ejercicio con respecto a la fecha en la que se realizó el mismo, para que vea su evolución respecto al tiempo. 

También la app ayudará a hacer un control de las metas que se proponga el usuario, es decir, si el usuario se propone bajar de peso o aumentar su rendimiento en x ejercicio, la app le recordará las metas que se autopropuso y le avisará cuando las cumpla. 

\section{Objetivos}
Los objetivos de este trabajo de fin de grado son:

\begin{itemize}
	\item Analizar algunas apps del mercado, así como sus capacidades, que ofrecen, su costo tanto para el usuario como para sus creadores y las valoraciones de los usuarios finales, para aclarar que es lo que buscan los usuarios en este tipo de apps y adaptar esas necesidades al software a desarrollar.
	\item Investigar sobre que herramientas usar para la implementación de la base de datos, backend, frontend y técnicas y metodologías para dar un desarrollo de calidad al software. Parte importante para ayudar a la sostenibilidad del sistema en el tiempo.
	\item Aprender a usar herramientas actuales y punteras para el desarrollo de aplicaciones.
	\item Desarrollar una aplicación multiplataforma que cubra todos los requisitos necesarios para dar soporte a la realización de entrenamientos y monitorización de rutinas de ejercicio físico.
\end{itemize}

\section{Estructura de la Memoria}

El 1º capítulo(\textbf{Introducción}), es una descripción sobre lo que se va a abordar y que ideas iniciales se tienen acerca del software que se va a desarrollar y algunos aspectos importantes del gremio en el que se usaría.

El 2º capítulo(\textbf{Estado del arte}), es un análisis de como está el ámbito sobre el que se va a desarrollar este trabajo, es decir, apps del mercado(sus prestaciones y funciones más importantes) y frameworks que se podrían emplear para el desarrollo, así como sus ventajas y desventajas.

El 3º capítulo,
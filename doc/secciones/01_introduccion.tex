\chapter{Introducción}

\section{Justificación}

En este TFG se va a desarrollar una aplicación que ayude a las personas a realizar ejercico físico de forma controlada y a monitorizar las metas que se van alcanzando.
Primero de todo,¿que me lleva a hacer este TFG? Para empezar, es verdad que hay muchas apps en los repositorios que están pensadas para ser usadas por usuarios de gimnasio, para hacer más fácil el seguimiento y evolución, no obstante, dichas aplicaciones suelen medir parámetros de forma muy general y de forma muy poco exacta, como las calorías quemadas.
% Lo siguiente es mejor ponerlo en el estado del arte, no en la justificación, además, no deberías mencionar aquí lo del reloj porque puede llevar a confusión sobre si tú % lo vas a usar o no:
% "Por ejemplo, pueden medir calorías quemadas basadas en las pulsaciones (si tienes un reloj inteligente que lo permita) pero este tipo de mediciones de calorías suelen 
%ser poco exactas, me baso en el siguiente estudio \cite{jerath2023future}, como añadido a esto no suele ser un dato de mucha relevancia al hacer datos ejericicios de %fuerza. 
Además la gente que suele hacer deporte de manera más informal o por hobbie no suele tener los conocimientos o dispositivos para hacer mediciones de calidad e interpretarlo correctamente, por lo que necesitarían una aplicación más sencilla tipo agenda para planificar y monitorizar sus ejercicios. 

Otra problemática que encontramos en las apps existentes suele ser la falta de accesibilidad, pensando en usuarios con algún tipo de discapacidad que quieran realizar ejercicios. Por ejemplo, encontramos que los scrolls abundan, hay eventos no controlados por el usuario, colores confusos para daltónicos, a veces hay ausencias de iconos asociados a botones y listas, y un amplio etcétera. 

Otras carencias que se observan es que no suelen incluir la funcionalidad de ver entrenamientos recomendados por otros. La gente suele buscarlos en redes sociales o videos que se encuentran en la red, y muchas veces en estos videos se ponen a dar rodeos para rascar más tiempo, así es como el usuario pierde tiempo para que a lo mejor no sea el entrenamiento que el andaba buscando. Lo ideal sería que los ejercicios estuvieran bien organizados en rutinas y que estas pudieran compartirse en la aplicación.

Me veo en la necesidad de hacer esta app para dotar de una herramienta simplificada, fácil de manejar, que permita compartir información entre usuarios de forma eficaz y accesible. Simplificada porque la información que manejo se trata de una forma fácil e intuitiva, con un poco de experiencia en el deporte se sabe qué significa cada parámetro a medir en un ejercicio,  % esto que hay a continuación no lo entiendo y la que no es tan fácil de interpretar, la interpreta el sistema por el usuario. 
Fácil de manejar, dado que el usuario solo se tendrá que preocupar de por ejemplo, contar cuantas veces levanta una pesa o indicar cuándo empieza y acaba de correr. Compartir información eficazmente, porque en una sección de la app habrá una parte formato red social, que permitirá tanto buscar usuarios que comparten entrenamientos, como los propios entrenamientos en si, acompañados de su descripción en la que el creador da una breve información acerca del entrenamiento. Accesible, porque se seguirán guías de diseño para facilitar el uso por usuarios de diversas capacidaddes.

Una vez explicado esto, ¿cómo se le dará soporte a los usuarios? Creando una app que le ayude de la siguiente forma: Ayudándole a planificar o usar rutinas de entrenamientos compuestas por series de ejercicios; Facilitándole la monitorización y supervisión de su realización, al poder introducir metas a alcanzar en parámetros, que pueden ser repeticiones/peso/distancia/tiempo, y que se pueden medir de forma independiente o al mismo tiempo. También guardando la cantidad de series de un ejercicio realizadas en un entrenamiento con sus respectivos parámetros, y contabilizandole al usuario el tiempo descansado, para facilitar su correcto entrenamiento. Además, si el usuario lo solicita se le enseñará un resumen de su rendimiento en cada ejercicio con respecto a la fecha en la que se realizó el mismo, para que vea su evolución respecto al tiempo. 

También la app ayudará a hacer un control de las metas que se proponga el usuario, es decir, si el usuario se propone bajar de peso o aumentar su rendimiento en x ejercicio, la app le recordará las metas que se autopropuso y le avisará cuando las cumpla. 

\section{Objetivos}

%Falta añadir un objetivo general, una frase que resuma lo que quieres que haga la app.

Los objetivos de este trabajo de fin de grado son:

\begin{itemize}
	\item Analizar algunas apps del mercado, así como sus características, qué ofrecen, su costo tanto para el usuario como para sus creadores y las valoraciones de los usuarios finales, para aclarar que es lo que buscan los usuarios en este tipo de apps y adaptar esas necesidades al software a desarrollar.
	\item Investigar sobre que herramientas usar para la implementación de la base de datos, backend, frontend y técnicas y metodologías para dar un desarrollo de calidad al software. Parte importante para ayudar a la sostenibilidad del sistema en el tiempo.
	\item Aprender a usar herramientas actuales y punteras para el desarrollo de aplicaciones.
	\item Desarrollar una aplicación multiplataforma que cubra todos los requisitos necesarios para dar soporte a la realización de entrenamientos y monitorización de rutinas de ejercicio físico.
\end{itemize}

\section{Estructura de la Memoria}

El 1º capítulo(\textbf{Introducción}), es una descripción sobre lo que se va a abordar y que ideas iniciales se tienen acerca del software que se va a desarrollar y algunos aspectos importantes del gremio en el que se usaría.

El 2º capítulo(\textbf{Estado del arte}), es un análisis de como está el ámbito sobre el que se va a desarrollar este trabajo, es decir, apps del mercado(sus prestaciones y funciones más importantes) y frameworks que se podrían emplear para el desarrollo, así como sus ventajas y desventajas.

El 3º capítulo,
